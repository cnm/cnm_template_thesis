\chapter{Implementation}
\label{chapter:implementation}

\section{Implementation Options}
Neste secção devem apresentar as opções de implementação que tinham ao vosso
dispor, avaliá-las e justificar a escolha que realizaram. Isto pode englobar:
\begin{itemize}
    \item Simuladores
    \item Linguagens e ambientes de programação
    \item Sistemas operativos
    \item Hardware
\end{itemize}
Fica sempre bem na avaliação colocar uma tabela com as características
pretendidas e as que são satisfeitas pelas várias opções (tipo catálogo com as
características dos automóveis). A escolha deve surgir naturalmente, com base na
opção que tem mais cruzes\ldots

\section{Architecture}
Nesta secção devem explicar como implementaram a vossa solução, apresentando
as simplificações que efectuaram, face ao modelo inicialmente previsto. As
simplificações devem ser devidamente justificadas. Se for possível, devem indicar
que estas não põem em causa as contribuições da tese.
Podem ainda descrever os principais problemas que tiveram e a forma como os
abordaram e resolveram.

Se estiverem a usar um simulador devem:

\begin{itemize}
    \item explicar o funcionamento do simulador
    \item explicar as alterações e modelos que desenvolveram no simulador e que permitem validar a vossa ideia
\end{itemize}

Se estiveram a desenvolver SW, sem simulador devem:
\begin{itemize}
    \item explicar os módulos, interfaces, estruturas de dados, etc\ldots
\end{itemize}

Sempre que possível, ilustrem a arquitectura com figuras que demonstrem a
evolução face à arquitectura da secção anterior. Isto é, usem as figuras anteriores
e façam as modificações necessárias à obtenção da arquitectura do protótipo.
\ldots
