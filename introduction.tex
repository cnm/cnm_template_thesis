\chapter{Introduction}\label{introduction}

\section{Background}
Nesta secção devem descrever a área em que a vossa tese se insere, de forma
a contextualizar o problema que vão resolver. No âmbito da área devem
identificar os principais problemas que existem.

Caso queiram usar acrónimos é assim \ac{IEEE}.
Proxima vez que usarem \ac{IEEE} já não expande.

\section{Proposed Solution}
Nesta secção devem identificar objectivamente o problema que vão resolver e o
tipo de solução que vão adoptar.

Exemplos de problemas que podem ser endereçados e tipos de soluções:

\begin{itemize}
\item Esta situação ainda não tem solução - > vou propor uma nova
\item Existem soluções para aspectos específicos, mas não existe uma que integre
os vários aspectos -> vou integrar as várias
\item Existem demonstrações matemáticas ou validações por simulação, mas
será que funcionam na realidade (quando as aproximações que se fizerem
deixarem de ser válidas) - > vou fazer um protótipo experimental
\item Existem demasiadas propostas de soluções, validadas em diferentes
condições –> vou fazer um estudo comparativo que as avalie sob diferentes
perspectivas
\item Existe uma boa solução, mas com um desempenho variável em função da
configuração- > vou fazer um estudo que permita verificar em que medida é
possível ajustar os parâmetros automaticamente às diferentes situações.
\item Existe uma teccnologia que ainda não está devidamente explorada -> vou
fazer um protótipo de ilustre o seu uso.
\end{itemize}

\section{Thesis Contribution}
Nesta secção devem identificar como é que a vossa solução vai contribuir para
resolver o problema.

\section{Outline}

This document describes the research and work developed and it is organized as follows:

\begin{itemize}
\item \textbf{Chapter \ref{introduction}} presents the motivation, background and proposed solution.
\item \textbf{Chapter \ref{relatedwork}} describes the previous work in the field.
\item \textbf{Chapter \ref{architecture}} describes the system requirements and the architecture of GBus.
\item \textbf{Chapter \ref{implementation}} describes the implementation of GBus and the technologies chosen.
\item \textbf{Chapter \ref{evaluation}} describes the evaluation tests performed and the corresponding results.
\item \textbf{Chapter \ref{conclusion}} summarizes the work developed and future work.
\end{itemize}

