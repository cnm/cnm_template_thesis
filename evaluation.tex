\chapter{Evaluation}\label{evaluation}
\section{Tests Objectives}
Nesta secção devem descrever os objectivos dos testes que realizaram, explicitando
que a razão pela qual os testes são relevantes face à validação das contribuições da
tese.

\section{Tests Scenarios}
Nesta secção devem descrever o cenário de teste, incluindo, por exemplo, a
definição da rede, o modelo de tráfego, as características de cada elemento....
A descrição deve ser feita, de forma a que os testes possam ser reprodutíveis.
Se os testes forem feitos em ambiente real devem ser descritas as características
dos equipamentos, memória, CPU, disco, SO, etc....

Devem também descrever as características das experiências, do ponto de vista
estatístico. Número de testes realizados, grandezas que vão ser medidas, formas de
medição dos valores, etc...

Sempre que possível, ilustrem o cenário de testes com figuras e com tabelas, que
descrevam sucintamente o modelo.

\section{Test Results}
Nesta secção devem apresentar os resultados dos testes, quer sobre a forma
de tabelas, quer sobre a forma de gráficos. As tabelas e os gráficos devem ser
apresentados e depois analisados, detalhadamente.
\ldots
